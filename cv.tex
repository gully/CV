\documentclass[10pt,letterpaper]{article}
\usepackage{graphicx}
\usepackage[colorlinks,urlcolor=blue]{hyperref}
\usepackage{color}
\usepackage{enumerate}
\usepackage{fancyhdr}
\usepackage{fontspec}
\usepackage{fontawesome5}
\usepackage[resetlabels]{multibib}

% Set margins.
\oddsidemargin=0.0in
\evensidemargin=0.0in
\textwidth=6.5in
\voffset=-0.125in
\headheight=0.0in
\topmargin=0.0in
\textheight=9.0in

% Define colors.
\definecolor{addresscolor}{rgb}{0.2,0.2,0.2}
\definecolor{datecolor}{rgb}{0.5,0.5,0.5}

% Turn off page numbering.
\pagestyle{fancy}
\fancyhf{}
\lfoot{\datestyle M.A. Gully-Santiago, PhD}
\cfoot{\datestyle \thepage}
\rfoot{\datestyle Aug 2023}
\renewcommand{\headrulewidth}{0.0pt}

% Change font.
\renewcommand{\familydefault}{\sfdefault}

% Set parameters and content.
\newcommand{\namestyle}{\Huge \scshape}
\newcommand{\deptstyle}{\footnotesize \sffamily \scshape}
\newcommand{\addressstyle}{\color{addresscolor}  \footnotesize \sffamily \upshape}
\newcommand{\datestyle}{\color{datecolor} \footnotesize \sffamily \upshape}

% Remove list labels.
\renewcommand{\labelitemi}{}
\renewcommand{\labelitemii}{}

% Define bibliographies.
\newcites{j,c}{\faUser ~ First Author Publications, \faUsers ~ Contributing Author Publications}


\begin{document}

\begin{flushleft}
    \namestyle Michael A. Gully-Santiago \\[0.3em]
    \addressstyle
    \faMapMarked* ~ 2515 Speedway, Stop C1400  \hfill \url{gully.github.io} ~ \faLink \\
    ~ ~ Austin, Texas 78712-1205  \hfill  \url{www.linkedin.com/in/gullys} ~ \faLinkedinIn \\
    \faPhone ~ (617) 842-5905 \hfill \url{github.com/gully} ~ \faGithub \\
    \faFlagUsa ~ US Citizen \hfill \textbf{igully@utexas.edu} ~ \faAt\\

\end{flushleft}

\small

\section*{\faToggleOn ~ Current Position}

\subparagraph{Environmental Data Scientist}
SeekOps, Inc. $\cdot$ Austin, TX $\cdot$ 01/2024--\emph{present}
\begin{itemize}
    \item Quantifying Greenhouse Gas Emissions with Tunable Diode Laser Absorption Spectroscopy
\end{itemize}

\section*{\faGraduationCap ~ Education}

\subparagraph{Ph. D., Astronomy}
University of Texas at Austin$\cdot$ Austin, TX $\cdot$ 8/2008--5/2015

\subparagraph{B. A., Astronomy \& Physics}
Boston University $\cdot$ Boston, MA $\cdot$ 9/2003--5/2007


\section*{\faTrophy ~ Awards}

\begin{itemize}
    \item 2019, Second Place, PyTorch Machine Learning / AI Summer Hackathon at Facebook HQ
    \item 2017, NASA Postdoctoral Program (NPP) Fellowship,  \emph{declined}
    \item 2016, Peking University Postdoctoral Defense High Pass
    \item 2014, University of Texas at Austin Department of Astronomy, David Benfield Memorial fellowship
    \item 2010-2013, NASA Graduate Student Research Program Fellowship, JPL Microdevices Lab
    \item 2010 \& 2011, University of Texas at Austin Dean's Prestigious Fellowship Supplement
    \item 2007 Boston University College Prize for Excellence in Astronomy
\end{itemize}

\section*{\faMoneyCheck* ~ Funding proposals as PI or Science PI }


\subparagraph{NASA TCAN}\mbox{} Theoretical Computational Astrophysics Networks\\
\emph{``Accelerating Substellar Atmosphere Spectral Inference with Machine Learning Technologies''} \\
\$1.5M; \emph{Selected for funding, Sept. 2023}; NNH22ZDA001N-TCAN\\
\faRocket ~~Principal Investigator: Gully-Santiago\\
Collaboration with AMNH (Institue PI: Jackie Faherty) \& NASA Ames (Institute PI: Natasha Batalha) \\

\subparagraph{NASA ADAP}\mbox{} Astrophysical Data Analysis Program\\
\emph{``Brown Dwarfs in High Definition: Confronting Substellar Atmosphere Models with the Keck-NIRSPEC Archive''} \\
\$380k; 2021 - 2024; NNH20ZDA001N-ADAP; \\
Administrative PI: Caroline Morley


\subparagraph{NASA TESS GI Cycle 4}\mbox{}\\
\emph{``A Systematic approach to quantifying starspot contrast with TESS and K2''}\\
\$69k; 2022 - 2023; NNH20ZDA001N-TESS; \\
Administrative PI: Caroline Morley


\section*{\faWrench ~ Research Experience and Technical Skills}

\subparagraph{Research Associate}
UT Austin Deptartment of Astronomy $\cdot$ Austin, TX $\cdot$ 09/2022--01/2025
\subparagraph{Research Fellow}
$\cdot$ 02/2020--09/2022
\begin{itemize}
    \item Member of Exoplanet Atmospheres Research Group led by Prof. Caroline Morley
\end{itemize}

\subparagraph{Support Scientist}
Kepler/K2 Guest Observer Office $\cdot$ Moffett Field, CA $\cdot$ 05/2017--01/2020

\subparagraph{Research Scientist}
\texttt{baeri.org} at NASA Ames Research Center $\cdot$ Moffett Field, CA $\cdot$ 02/2017--05/2017
\begin{itemize}
    \item  \textbf{Forward modeling Keck and IRTF spectra:} Analysis of low resolution near-IR spectroscopy of young stars and brown dwarfs with collaborators T. Greene and M. Marley
\end{itemize}

\subparagraph{Postdoctoral Researcher}
Kavli Institute for Astronomy and Astrophysics $\cdot$ Beijing, China $\cdot$ 10/2015--10/2016
\begin{itemize}
    \item  \textbf{Forward modeling IGRINS spectra:} Analysis of high resolution, high bandwidth near-IR spectroscopy of young stars with collaborator G. Herczeg
\end{itemize}

\subparagraph{Si diffractive optics group, Dept. of Astronomy}
University of Texas at Austin $\cdot$ Austin, TX $\cdot$ 9/2008--6/2014
\begin{flushright}
    Microelectronics Research Center $\cdot$ Austin, TX $\cdot$ 9/2008--6/2013

    Center for Nano and Molecular Science $\cdot$ Austin, TX $\cdot$ 9/2008--9/2013
\end{flushright}


\subparagraph{E-beam group, Microdevices Laboratory}
NASA Jet Propulsion Lab $\cdot$ Pasadena, CA $\cdot$ 9/2010--9/2013

\subparagraph{Guest Observer, Magellan Telescope}
Las Campanas Observatory $\cdot$ La Serena, Chile $\cdot$ 2010--2012


\section*{\faSlideshare ~ Talks and Conference Participation}

Select presentations have \texttt{YouTube} videos (\faYoutube) or \texttt{SpeakDeck} slides (\faSpeakerDeck) available.

\begin{itemize}


    \item Talk, \href{https://speakerdeck.com/gully/a-large-and-variable-leading-tail-of-helium-in-a-hot-saturn-undergoing-runaway-inflation}{\faSpeakerDeck} Large Leading Tail of Helium in a Hot Saturn Undergoing Runaway Inflation, Towards Other Earths, 7/2023
    \item Talk, \href{https://speakerdeck.com/gully/blase-an-interpretable-transfer-learning-approach-to-cool-star-echelle-spectroscopy}{\faSpeakerDeck} Interpretable Transfer Learning for Cool Star Spectroscopy, Machine Learning Cool Stars 21, 7/2022
    \item Talk, Technologies for Precision Stellar Activity, Penn State CEHW Seminar, 4/2022
    \item Talk, Growing an ecosystem of spectral investigative tools, UT Austin, 9/2021
    \item Talk, \href{https://youtu.be/0yLgE_8YsIM?t=130}{\faYoutube} Condensate cloud modulation in IGRINS and TESS, TESS Science Conference, 8/2021
    \item Talk, \href{https://youtu.be/ME7kSjPe7mM}{\faYoutube} Applying Probabilistic Inference to Astronomical Spectroscopy, SciPy Conferece, 7/2020
    \item Talk, \href{https://speakerdeck.com/gully/frontiers-in-forward-modeling-substellar-atmospheres}{\faSpeakerDeck} Frontiers in forward modeling substellar atmospheres, UT Austin, 10/2020
    \item Talk, \href{https://speakerdeck.com/gully/know-thy-planet-know-thy-starspots}{\faSpeakerDeck} Know Thy Planet Know Thy Starspots, Exoplanet Spectroscopy e-Workshop, 10/2019
    \item Talk, Precision Stellar Activity, U. Arizona, Tucson, AZ, 1/2019
    \item Talk, \href{https://speakerdeck.com/gully/k2-and-igrins-constrain-starspot-filling-factors-and-temperatures}{\faSpeakerDeck} Kepler/K2 and IGRINS constrain starspots, AAS233, Seattle, WA, 1/2019
    \item Talk, Precision Stellar Activity, UT Austin, Austin, TX, 11/2018
    \item Talk, \href{https://speakerdeck.com/gully/k2-and-high-resolution-near-ir-spectroscopy}{\faSpeakerDeck} Measuring starspot physical properties, PLATO-ESP, Marseille, France, 10/2018
    \item Poster, Physical properties of starspots, Cool Stars 20, Boston, MA, 7/2018
    \item Talk, \href{https://speakerdeck.com/gully/gpus-for-astronomy-data}{\faSpeakerDeck} GPUs for Astronomy Data, NVIDIA Endeavor Research Center, Santa Clara, CA, 4/2018,
    \item Poster, Physical properties of starspots, NASA Ames Space Science Jamboree, Moffett Field, CA, 4/2018
    \item Talk, Starspots Confound Planet Transit Spectra, Bay Area Exoplanet Meeting, Moffett Field, CA, 3/2018
    \item Lightning Talk, Starspots, UC Berkeley Astronomy Lunch Talk, Berkeley, CA, 2/2018,
    \item Talk, Starspots with K2 and IGRINS, K2 Dwarf Stars and Clusters Workshop, Boston, MA, 1/2018
    \item Poster, Physical properties of starspots, Know Thy Star Know Thy Planet, Pasadena, CA, 10/2017
    \item Tutorial, The Starfish Spectral Inference Framework, Other Worlds Laboratory, UCSC, CA, 7/2017
    \item Talk, Physical properties of starspots, Kepler/K2 Science Conference IV, Moffett Field, CA, 6/2017
    \item Talk, Fundamental properties of youngs stars, KIPAC, Stanford University, CA, 3/2017
    \item Talk, Abolute stellar ages and planet formation timescales, Bay Area Exoplanets, NASA Ames, CA, 3/2017
    \item Talk, \href{https://speakerdeck.com/gully/measuring-fundamental-properties-of-young-stars}{\faSpeakerDeck} Measuring Fundamental Properties of Young Stars, Columbia U., NYC, NY, 11/2016
    \item Talk, Measuring Fundamental Properties of Young Stars, Simons CCA, NYC, NY, 11/2016
    \item Talk, Measuring Fundamental Properties of Young Stars, Boston U., Boston, MA 11/2016
    \item Talk, Measuring Fundamental Properties of Young Stars, KIAA Beijing, China, 9/2016
    \item Talk, Python for astronomy, Beijing Python Meetup, China, 8/2016
    \item Poster, Measurement of starspot properties, Cool Stars 19, Uppsala, Sweden 6/2016
    \item Talk, High Resolution Spectroscopy with IGRINS, Seoul, Korea, 11/2015
    \item Attendee, Astro Data Hack Week, Seattle, WA, 9/2014
    \item Poster, SPIE Astronomical Telescopes and Instrumentation, Montreal, QC, 6/2014
    \item Poster, PPVI, Heidelberg, Germany, 7/2013
    \item Talk, Star Formation Lunch, Jet Propulsion Lab, Pasadena, CA, 6/2013
    \item Poster, Award winner- $3^{rd}$/45, Nano Night, Center for Nano- and Molecular Science, Austin, TX, 3/2013
    \item Poster, McDonald Observatory Board of Visitors meeting, Austin, TX, 2/2013
    \item Invited Talk, SPIE Astronomical Telescopes and Instrumentation, Amsterdam, NL, July, 2012
    \item Poster, Cool Stars 17, Barcelona, Spain, June 2012
    \item Attendee, American Astronomical Society meeting, Austin, TX, Jan, 2012
    \item Talk, Very Low Mass Stars and Brown Dwarfs, ESO, Garching, Germany, 10/2011
    \item Attendee, National Society of Black and Hispanic Physicists, Austin, TX, 9/2011
    \item Poster, Cool Stars 16, Seattle, WA, 9/2010
    \item Poster, SPIE Astronomical Telescopes and Instrumentation, San Diego, CA, 6/2010
\end{itemize}


\section*{\faChalkboardTeacher ~ Teaching, Service, Leadership}

\subparagraph{Students mentored}
\begin{itemize}
    \item Sujay Shankar; Undergrad $\cdot$ UT Austin $\cdot$ 2022-\emph{present}
    \item Ryan Hartung; Undergrad $\cdot$ UT Austin $\cdot$ Summer 2022
    \item Jiayi Cao; Undergrad $\cdot$ UT Austin $\cdot$ 2022
    \item Erica Sawczynec; Grad Student (\emph{consulting role})$\cdot$ UT Austin $\cdot$ 2022
    \item Emily Lubar; Grad Student (\emph{consulting role})$\cdot$ UT Austin $\cdot$ 2022
    \item Joel Burke; Undergrad (\emph{consulting role})$\cdot$ UT Austin $\cdot$ 2021
    \item Diana Gonzalez-Argueta; TAURUS Program Undergrad $\cdot$ UT Austin $\cdot$ Summer 2021
    \item Karina Kimani-Stewart; TAURUS Program Undergrad$\cdot$ UT Austin $\cdot$ Summer 2021
    \item Aishwarya Ganesh; Undergrad $\cdot$ UT Austin $\cdot$  2020--2022
    \item Jessica Luna; Grad Student (\emph{consulting role}) $\cdot$ UT Austin $\cdot$  2020--2022
    \item Sheila Sagear; NASA Summer Undergrad Intern $\cdot$ Kepler/K2 Science Center $\cdot$ Summer 2018
    \item Amanda Turbyfill; Undergrad $\cdot$ UT Austin $\cdot$ 2013--2014
\end{itemize}


\subparagraph{Hackathon Organizer}
UT Austin Astronomy Hackathon$\cdot$ Austin, TX $\cdot$ 2015, 2022

\subparagraph{Statistical computing tutorial leader}
Kavli Institute for Astronomy \& Astrophysics $\cdot$ Beijing, China $\cdot$ 2015--2016

\subparagraph{Graduate Student Representative}
University of Texas at Austin Department of Astronomy $\cdot$ 6/2011--6/2012

\subparagraph{Faculty member}
Clay Center Observatory at the Dexter \& Southfield Schools  $\cdot$ Brookline, MA $\cdot$ 6/2007--6/2008

\subparagraph{Adult and continuing education instructor}
Brookline Adult Education  $\cdot$ Brookline, MA $\cdot$ 6/2005--6/2008

\subparagraph{Night lab teaching assistant}
Boston University $\cdot$ Boston, MA $\cdot$ 2006--2007

\section*{\faYoutube ~ Public Outreach and Media Appearances}

\subparagraph{Screencast producer}
\begin{itemize}
    \item YouTube \texttt{lightkurve} tutorials $\cdot$ 2018--2019
\end{itemize}


\subparagraph{Podcast Appearances}
\begin{itemize}
    \item Blue Dot Podcast: ``The K2 Mission'', NCPR, 6/2018
    \item ``Discovery and characterization of brown dwarfs'', KVRX, 91.7FM $\cdot$ Austin, TX$\cdot$ 12/2012
\end{itemize}

\subparagraph{Podcast Host, \emph{They Blinded Me with Science} }
KVRX, 91.7FM $\cdot$ Austin, TX$\cdot$ 5/2013--5/2014
\begin{itemize}
    \item Produced and/or co-hosted 30 original science podcasts, with seed funding from UT College of Natural Sciences
\end{itemize}

\subparagraph{Public talks and appearances}
\begin{itemize}
    \item Talk, ``How stars and planets form'', Astronomy on Tap Bay Area, San Jose, CA, 2/2018
    \item \texttt{Nightlife} Public Engagement, Cal Academy of Sciences, San Francisco, CA, 2017 \& 2018
    \item \textbf{Invited talk}, McDonald Observatory Board of Visitors meeting, Austin, TX, 2/2012
    \item Science Under the Stars, Brackenridge Field Lab, Austin, TX, 12/2012
\end{itemize}

\subparagraph{Interactive museum-style educational installation}
Department of Astronomy $\cdot$ Austin, TX$\cdot$ 7/2013--9/2014

\section*{ \faSuperscript ~ Unique coursework or independent study}

\subparagraph{Statistical Modeling II, Prof. James Scott}
Statistics Department$\cdot$ 1/2014--5/2014

\subparagraph{Statistics, Data Mining and Machine Learning in Astronomy}
Independent study$\cdot$ 1/2014--8/2014

\section*{ \faLaptopCode ~ Computer Skills}

\begin{itemize}
    \item \textbf{Creator:} \texttt{muler}, \texttt{gollum}, \texttt{blas\'e}, \texttt{ynot}
    \item \textbf{Maintainer:} \texttt{Starfish}, \texttt{lightkurve}, \texttt{telfit}
    \item \faPython, \faTerminal, \faGit, \faGithub, \LaTeX, \faApple, \faLinux, \faWindows, \texttt{bokeh}, \texttt{conda}, \texttt{IDL}, \texttt{PyTorch}, \texttt{JAX}
    \item NASA Advanced Supercomputing (NAS) High End Computing Capability (HECC) \emph{Pleiades} 2018--2020
    \item Texas Advanced Computing Center (TACC): \emph{Maverick} 2015, \emph{Frontera} 2020 -- present
\end{itemize}


% Publications.
\bibliographystylej{IEEEtran}
\bibliographystylec{IEEEtran}


\nocitej{2023arXiv230708959G}
\nocitej{2022ApJ...941..200G}
\nocitej{2022JOSS....7.4302G}
\nocitej{2017ApJ...836..200G}
\nocitej{2015ApOpt..5410177G}
\nocitej{2014SPIE.9151E..5KG}
\nocitej{2012SPIE.8450E..2SG}
\nocitej{2011ASPC..448..633G}
\nocitej{2010SPIE.7739E..3SG}
\nocitec{2023SciA....9F8736Z}
\nocitec{2023ApJ...950...99S}
\nocitec{2023AJ....165...46T}
\nocitec{2022AJ....164...59L}
\nocitec{2022AJ....164...59L}
\nocitec{2022ApJ...927..222L}
\nocitec{2022ApJ...925...75P}
\nocitec{2022ApJ...925....5G}
\nocitec{2021ApJ...923..167W}
\nocitec{2021MNRAS.507.3125A}
\nocitec{2021ApJ...921...53L}
\nocitec{2021AJ....162..213F}
\nocitec{2020SPIE11451E..5IL}
\nocitec{2020AJ....160..219F}
\nocitec{2020MNRAS.498...33R}
\nocitec{2020AJ....159..112M}
\nocitec{2019MNRAS.490.5551R}
\nocitec{2019ApJ...882...49L}
\nocitec{2019MNRAS.486..453L}
\nocitec{2019A&A...622A..75L}
\nocitec{2019ApJ...870L...1D}
\nocitec{2019ApJ...870...13S}
\nocitec{2019ApJ...870...12L}
\nocitec{2018ApJ...869...17L}
\nocitec{2018ApJ...868..143G}
\nocitec{2018ApJ...862...85G}
\nocitec{2018arXiv180308708A}
\nocitec{2017PASP..129f5004D}
\nocitec{2016ApJ...831..133H}
\nocitec{2016SPIE.9908E..0CM}
\nocitec{2014SPIE.9152E..02K}
\nocitec{2014SPIE.9151E..1GB}
\nocitec{2014SPIE.9147E..22J}
\nocitec{2014SPIE.9147E..1DP}
\nocitec{2013AN....334..159J}
\nocitec{2012SPIE.8550E..1BH}
\nocitec{2010SPIE.7739E..4LW}
\nocitec{2010SPIE.7735E..2KL}
\nocitec{2010SPIE.7735E..1MY}
\nocitec{2010SPIE.7731E..0CG}
\bibliographyj{gullyCV}
\bibliographyc{gullyCV}




\end{document}
